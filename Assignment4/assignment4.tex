\documentclass[journal,12pt,twocolumn]{IEEEtran}
\documentclass[addpoints]{exam}
\usepackage{setspace}
\usepackage{gensymb}
\singlespacing
\usepackage[cmex10]{amsmath}

\usepackage{amsthm}

\usepackage{mathrsfs}
\usepackage{txfonts}
\usepackage{stfloats}
\usepackage{bm}
\usepackage{cite}
\usepackage{cases}
\usepackage{subfig}

\usepackage{longtable}
\usepackage{multirow}

\usepackage{enumitem}
\usepackage{mathtools}
\usepackage{steinmetz}
\usepackage{tikz}
\usepackage{circuitikz}
\usepackage{verbatim}
\usepackage{tfrupee}
\usepackage[breaklinks=true]{hyperref}
\usepackage{graphicx}
\usepackage{tkz-euclide}

\usetikzlibrary{calc,math}
\usepackage{listings}
    \usepackage{color}                                            %%
    \usepackage{array}                                            %%
    \usepackage{longtable}                                        %%
    \usepackage{calc}                                             %%
    \usepackage{multirow}                                         %%
    \usepackage{hhline}                                           %%
    \usepackage{ifthen}                                           %%
    \usepackage{lscape}     
\usepackage{multicol}
\usepackage{chngcntr}
\usepackage{hyperref}

\DeclareMathOperator*{\Res}{Res}

\renewcommand\thesection{\arabic{section}}
\renewcommand\thesubsection{\thesection.\arabic{subsection}}
\renewcommand\thesubsubsection{\thesubsection.\arabic{subsubsection}}

\renewcommand\thesectiondis{\arabic{section}}
\renewcommand\thesubsectiondis{\thesectiondis.\arabic{subsection}}
\renewcommand\thesubsubsectiondis{\thesubsectiondis.\arabic{subsubsection}}


\hyphenation{op-tical net-works semi-conduc-tor}
\def\inputGnumericTable{}                                 %%

\lstset{
%language=C,
frame=single, 
breaklines=true,
columns=fullflexible
}

\begin{document}

\newcommand{\BEQA}{\begin{eqnarray}}
\newcommand{\EEQA}{\end{eqnarray}}
\newcommand{\define}{\stackrel{\triangle}{=}}
\bibliographystyle{IEEEtran}
\raggedbottom
\setlength{\parindent}{0pt}
\providecommand{\mbf}{\mathbf}
\providecommand{\pr}[1]{\ensuremath{\Pr\left(#1\right)}}
\providecommand{\qfunc}[1]{\ensuremath{Q\left(#1\right)}}
\providecommand{\sbrak}[1]{\ensuremath{{}\left[#1\right]}}
\providecommand{\lsbrak}[1]{\ensuremath{{}\left[#1\right.}}
\providecommand{\rsbrak}[1]{\ensuremath{{}\left.#1\right]}}
\providecommand{\brak}[1]{\ensuremath{\left(#1\right)}}
\providecommand{\lbrak}[1]{\ensuremath{\left(#1\right.}}
\providecommand{\rbrak}[1]{\ensuremath{\left.#1\right)}}
\providecommand{\cbrak}[1]{\ensuremath{\left\{#1\right\}}}
\providecommand{\lcbrak}[1]{\ensuremath{\left\{#1\right.}}
\providecommand{\rcbrak}[1]{\ensuremath{\left.#1\right\}}}
\providecommand{\prlbrak}[1]{\ensuremath{\Pr\left(#1\right.}}

\theoremstyle{remark}
\newtheorem{rem}{Remark}
\newcommand{\sgn}{\mathop{\mathrm{sgn}}}
\providecommand{\abs}[1]{\vert#1\vert}
\providecommand{\res}[1]{\Res\displaylimits_{#1}} 
\providecommand{\norm}[1]{\lVert#1\rVert}
%\providecommand{\norm}[1]{\lVert#1\rVert}
\providecommand{\mtx}[1]{\mathbf{#1}}
\providecommand{\mean}[1]{E[ #1 ]}
\providecommand{\fourier}{\overset{\mathcal{F}}{ \rightleftharpoons}}
%\providecommand{\hilbert}{\overset{\mathcal{H}}{ \rightleftharpoons}}
\providecommand{\system}{\overset{\mathcal{H}}{ \longleftrightarrow}}
	%\newcommand{\solution}[2]{\textbf{Solution:}{#1}}
\newcommand{\solution}{\noindent \textbf{Solution: }}
\newcommand{\cosec}{\,\text{cosec}\,}
\providecommand{\dec}[2]{\ensuremath{\overset{#1}{\underset{#2}{\gtrless}}}}
\newcommand{\myvec}[1]{\ensuremath{\begin{pmatrix}#1\end{pmatrix}}}
\newcommand{\mydet}[1]{\ensuremath{\begin{vmatrix}#1\end{vmatrix}}}
\numberwithin{equation}{subsection}
\makeatletter
\@addtoreset{figure}{problem}
\makeatother
\let\StandardTheFigure\thefigure
\let\vec\mathbf
\renewcommand{\thefigure}{\theproblem}
\def\putbox#1#2#3{\makebox[0in][l]{\makebox[#1][l]{}\raisebox{\baselineskip}[0in][0in]{\raisebox{#2}[0in][0in]{#3}}}}
     \def\rightbox#1{\makebox[0in][r]{#1}}
     \def\centbox#1{\makebox[0in]{#1}}
     \def\topbox#1{\raisebox{-\baselineskip}[0in][0in]{#1}}
     \def\midbox#1{\raisebox{-0.5\baselineskip}[0in][0in]{#1}}
\vspace{3cm}
\title{AI1103-Assignment 4}
\author{Name : Umesh Kalvakuntla     Roll No. : CS20BTECH11024}
\maketitle
\newpage
\bigskip
\renewcommand{\thefigure}{\theenumi}
\renewcommand{\thetable}{\theenumi}

Download latex-tikz codes from 
%
\begin{lstlisting}
https://github.com/Umesh-k26/AI-1103/blob/main/Assignment4/assignment4.tex
\end{lstlisting}
and python codes from 
\begin{lstlisting}
https://github.com/Umesh-k26/AI-1103/tree/main/Assignment4/codes
\end{lstlisting}
\section*{UGC/MATH 2019, Q.50}
\question Let $X_1, X_2, X_3, X_4, X_5$ be \textit{i.i.d.} random variables having a continuous distribution function. 
Then 
\begin{equation*}
    \begin{split}
        \pr{X_1 > X_2 > X_3 > X_4 > X_5 | X_1 &= max(X_1, X_2, \\
        & \qquad X_3, X_4, X_5)}
    \end{split}
\end{equation*}
equals \rule{1cm}{0.2mm}.

\begin{enumerate}
     \item $\dfrac{1}{4}$\\
     \item $\dfrac{1}{5}$\\
     \item $\dfrac{1}{4!}$\\
     \item $\dfrac{1}{5!}$
\end{enumerate} 

\section*{Solution}
Since $X_1$, $X_2$, $X_3$, $X_4$ and $X_5$ are identical and independently distributed random variables, 
they can be represented by a single random variable $X$.\\
Let \[\{x_1, x_2, x_3, x_4, x_5\} \in X\]
Required probability,
\begin{equation}\label{eq1}
    \begin{split}
        = \prlbrak{X_1 > X_2 > X_3 > X_4 > X_5 | X_1 &= max(X_1, X_2,} \\
        & \qquad \rbrak{X_3, X_4, X_5)}
    \end{split}
\end{equation}
\begin{equation}
    \begin{split}
        = \prlbrak{x_1>x_2>x_3>x_4>x_5 | x_1 &= max(x_1, x_2,}  \\
        & \qquad \rbrak{x_3, x_4, x_5)}
    \end{split}
\end{equation}
Required probability, 
\begin{align}
    &= \pr{x_2>x_3>x_4>x_5}\\
    \begin{split}\label{eq4}
        &= \pr{x_2 = max(x_2, x_3, x_4, x_5)} \\
        & \qquad \times \pr{x_3 = max(x_3, x_4, x_5)} \\
        & \qquad \times \pr{x_4 = max(x_4, x_5)}
    \end{split}
\end{align}
\\
Since $x_2, x_3, x_4, x_5$ are distinct, only one of them can be maximum. And the probability that $x_2$ being maximum is $\dfrac{1}{4}$.\\\\
Similarly, the probabilities of other two terms of the equation \eqref{eq4} are $\dfrac{1}{3}$ and $\dfrac{1}{2}$ respectively.\\\\
$\therefore$ Required probability is,
\begin{align}
    &= \frac{1}{4}\times \frac{1}{3} \times \frac{1}{2} \\
    &= \frac{1}{4!}
\end{align}
\\The correct answer is \textbf{Option 3}.
\end{document}
