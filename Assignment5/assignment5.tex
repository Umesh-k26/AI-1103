\documentclass[journal,12pt,twocolumn]{IEEEtran}
\documentclass[addpoints]{exam}
\usepackage{setspace}
\usepackage{gensymb}
\singlespacing
\usepackage[cmex10]{amsmath}

\usepackage{amsthm}

\usepackage{mathrsfs}
\usepackage{txfonts}
\usepackage{stfloats}
\usepackage{bm}
\usepackage{cite}
\usepackage{cases}
\usepackage{subfig}

\usepackage{longtable}
\usepackage{multirow}

\usepackage{enumitem}
\usepackage{mathtools}
\usepackage{steinmetz}
\usepackage{tikz}
\usepackage{circuitikz}
\usepackage{verbatim}
\usepackage{tfrupee}
\usepackage[breaklinks=true]{hyperref}
\usepackage{graphicx}
\usepackage{tkz-euclide}

\usetikzlibrary{calc,math}
\usepackage{listings}
    \usepackage{color}                                            %%
    \usepackage{array}                                            %%
    \usepackage{longtable}                                        %%
    \usepackage{calc}                                             %%
    \usepackage{multirow}                                         %%
    \usepackage{hhline}                                           %%
    \usepackage{ifthen}                                           %%
    \usepackage{lscape}     
\usepackage{multicol}
\usepackage{chngcntr}
\usepackage{hyperref}

\DeclareMathOperator*{\Res}{Res}

\renewcommand\thesection{\arabic{section}}
\renewcommand\thesubsection{\thesection.\arabic{subsection}}
\renewcommand\thesubsubsection{\thesubsection.\arabic{subsubsection}}

\renewcommand\thesectiondis{\arabic{section}}
\renewcommand\thesubsectiondis{\thesectiondis.\arabic{subsection}}
\renewcommand\thesubsubsectiondis{\thesubsectiondis.\arabic{subsubsection}}


\hyphenation{op-tical net-works semi-conduc-tor}
\def\inputGnumericTable{}                                 %%

\lstset{
%language=C,
frame=single, 
breaklines=true,
columns=fullflexible
}

\begin{document}

\newcommand{\BEQA}{\begin{eqnarray}}
\newcommand{\EEQA}{\end{eqnarray}}
\newcommand{\define}{\stackrel{\triangle}{=}}
\bibliographystyle{IEEEtran}
\raggedbottom
\setlength{\parindent}{0pt}
\providecommand{\mbf}{\mathbf}
\providecommand{\pr}[1]{\ensuremath{\Pr\left(#1\right)}}
\providecommand{\qfunc}[1]{\ensuremath{Q\left(#1\right)}}
\providecommand{\sbrak}[1]{\ensuremath{{}\left[#1\right]}}
\providecommand{\lsbrak}[1]{\ensuremath{{}\left[#1\right.}}
\providecommand{\rsbrak}[1]{\ensuremath{{}\left.#1\right]}}
\providecommand{\brak}[1]{\ensuremath{\left(#1\right)}}
\providecommand{\lbrak}[1]{\ensuremath{\left(#1\right.}}
\providecommand{\rbrak}[1]{\ensuremath{\left.#1\right)}}
\providecommand{\cbrak}[1]{\ensuremath{\left\{#1\right\}}}
\providecommand{\lcbrak}[1]{\ensuremath{\left\{#1\right.}}
\providecommand{\rcbrak}[1]{\ensuremath{\left.#1\right\}}}
\theoremstyle{remark}
\newtheorem{rem}{Remark}
\newcommand{\sgn}{\mathop{\mathrm{sgn}}}
\providecommand{\abs}[1]{\vert#1\vert}
\providecommand{\res}[1]{\Res\displaylimits_{#1}} 
\providecommand{\norm}[1]{\lVert#1\rVert}
%\providecommand{\norm}[1]{\lVert#1\rVert}
\providecommand{\mtx}[1]{\mathbf{#1}}
\providecommand{\mean}[1]{E[ #1 ]}
\providecommand{\fourier}{\overset{\mathcal{F}}{ \rightleftharpoons}}
%\providecommand{\hilbert}{\overset{\mathcal{H}}{ \rightleftharpoons}}
\providecommand{\system}{\overset{\mathcal{H}}{ \longleftrightarrow}}
	%\newcommand{\solution}[2]{\textbf{Solution:}{#1}}
\newcommand{\solution}{\noindent \textbf{Solution: }}
\newcommand{\cosec}{\,\text{cosec}\,}
\providecommand{\dec}[2]{\ensuremath{\overset{#1}{\underset{#2}{\gtrless}}}}
\newcommand{\myvec}[1]{\ensuremath{\begin{pmatrix}#1\end{pmatrix}}}
\newcommand{\mydet}[1]{\ensuremath{\begin{vmatrix}#1\end{vmatrix}}}
\numberwithin{equation}{subsection}
\makeatletter
\@addtoreset{figure}{problem}
\makeatother
\let\StandardTheFigure\thefigure
\let\vec\mathbf
\renewcommand{\thefigure}{\theproblem}
\def\putbox#1#2#3{\makebox[0in][l]{\makebox[#1][l]{}\raisebox{\baselineskip}[0in][0in]{\raisebox{#2}[0in][0in]{#3}}}}
     \def\rightbox#1{\makebox[0in][r]{#1}}
     \def\centbox#1{\makebox[0in]{#1}}
     \def\topbox#1{\raisebox{-\baselineskip}[0in][0in]{#1}}
     \def\midbox#1{\raisebox{-0.5\baselineskip}[0in][0in]{#1}}
\vspace{3cm}
\title{AI1103-Assignment 5}
\author{Name : Umesh Kalvakuntla     Roll No. : CS20BTECH11024}
\maketitle
\newpage
\bigskip
\renewcommand{\thefigure}{\theenumi}
\renewcommand{\thetable}{\theenumi}

Download latex-tikz codes from 
%
\begin{lstlisting}
https://github.com/Umesh-k26/AI-1103/blob/main/Assignment4/assignment4.tex
\end{lstlisting}
and python codes from 
\begin{lstlisting}
https://github.com/Umesh-k26/AI-1103/tree/main/Assignment4/codes
\end{lstlisting}
\section*{CSIR UGC NET EXAM (June 2012), Q.52}
\section*{Question} Suppose $X$ and $Y$ are independent random variables where $Y$ is symmetric about 0. Let
$U= X+Y$ and $V= X-Y$. Then
\begin{enumerate}
    \item $U$ and $V$ are always independent.
    \item $U$ and $V$ have the same distribution.
    \item $U$ is always symmetric about 0.
    \item $V$ is always symmetric about 0.
\end{enumerate}

\section*{Solution}
$Y$ is symmetric about $0$,
\begin{equation} \label{eq1}
    \implies f_Y(-y) = f_Y(y)
\end{equation}
\begin{align}
    F_U(u) &= \pr{U\leq u} \\
           &= \pr{X+Y \leq u}\\
           &= \pr{X \leq u-Y} \\
    F_U(u) &= \int_{-\infty}^{\infty} f_Y(y) \int_{-\infty}^{u-y} f_X(x) \,dx \, dy \label{eq5}\\
\intertext{differentiating equation \eqref{eq5} gives,}
    f_U(u) &= \int_{-\infty}^{\infty} f_Y(y) f_X(u-y) \, dy\\
\begin{split}
    f_U(u) &= \int_{-\infty}^{\infty} f_Y(-y) f_X(u+y) \, dy \\
           &\qquad \brak{\because \int_a^b f(x) \,dx = \int_a^b f(a+b-x) \,dx}
\end{split}
\intertext{from \eqref{eq1}, }
     f_U(u) &= \int_{-\infty}^{\infty} f_Y(y) f_X(u+y) \, dy \label{eq8}
\end{align}

\begin{align}
    F_V(v) &= \pr{V \leq v} \\
           &= \pr{X-Y \leq v} \\
           &= \pr{X \leq v+Y} \\
    F_V(v) &= \int_{-\infty}^{\infty} f_Y(y) \int_{-\infty}^{v+y} f_X(x) \,dx \,dy \label{eq12}\\
\intertext{differentiating equation \eqref{eq12} gives,}
    f_V(v) &= \int_{-\infty}^{\infty} f_Y(y) f_X(v+y) \, dy \label{eq13}
\end{align}

From \eqref{eq8} and \eqref{eq13}, $U$ and $V$ have same distribution.\\\\
$\therefore$  The correct answer is \textbf{Option 2}.
\end{document}
